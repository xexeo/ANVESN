\documentclass[a4paper,12pt]{book}
\usepackage[utf8]{inputenc}
\usepackage{geometry}
\geometry{a4paper, margin=1in}
\setcounter{secnumdepth}{3}
\setcounter{tocdepth}{2}

\usepackage{modernrules}

\title{Manual de Regras Simples para RPG: ANVESN}
\author{Baseado em The Laundry Files e Call of Cthulhu}
\date{}

\begin{document}

\maketitle

\tableofcontents

\chapter{Introdução}

A Agência Nacional de Vigilância de Eventos Sobrenaturais (ANVESN) é uma organização super secreta, destinada ao monitoramento e contenção de fenômenos sobrenaturais e ameaças extraordinárias à segurança nacional. Inspirado por sistemas simples e intuitivos, este RPG oferece uma experiência acessível, com mecânicas de insanidade e habilidades agrupadas em categorias.

\section{Objetivo do Jogo}

Os jogadores representam agentes da ANVESN, enfrentando entidades sobrenaturais, investigando atividades paranormais e lidando com os riscos físicos e psicológicos que essas missões impõem. Os jogadores são sempre personagens humanos.

\chapter{Criação de Personagem}

\section{Atributos Básicos: Definindo a Capacidade Física}

Nesse jogo, toda a capacidade ligada ao intelecto e emoções é responsabilidade do jogador. Os atributos físicos, porém, são do personagem. Eles são gerados com os 3 maiores resultados em uma rolagem de 4d6 (4D6-L). 

Os atributos básicos são:

\begin{itemize}
\item \textbf{Força}: Capacidade física para atividades intensas.
\item \textbf{Destreza}: Reflexos, precisão e agilidade.
\item \textbf{Constituição}: Resistência e saúde geral.
\item \textbf{Beleza}: Aparência física, sem poderes de carisma.
\end{itemize}

\section{NPCs}

Os NPCs possuem características físicas e emocionais básicas:
\begin{itemize}
    \item \textbf{Força, Destreza, Constituição e Beleza}: Correspondem aos atributos físicos.
    \item \textbf{Inteligência}: Capacidade de dedução e conhecimento.
    \item \textbf{Empatia}: Controle emocional e compreensão de outros.
\end{itemize}

\section{Formação e Habilidades}

Cada personagem escolhe uma formação profissional, que determinará suas habilidades principais e complementares. Alguns exemplos são: 

\begin{itemize}
    \item \textbf{Cientista}: Conhecimento teórico e técnico.
    \item \textbf{Investigador}: Percepção e raciocínio.
    \item \textbf{Combatente}: Foco em habilidades de combate.
    \item \textbf{Operador de Campo}: Habilidades práticas.
\end{itemize}

Cada personagem escolhe \textbf{5 habilidades principais} e \textbf{5 complementares}. Exemplos são:

\begin{itemize}
    \item \textbf{Habilidades Físicas}: Luta, Esquiva, Atletismo
    \item \textbf{Habilidades de Conhecimento}: Ocultismo, Ciências, História
    \item \textbf{Habilidades Práticas}: Investigação, Tecnologia, Primeiros Socorros
\end{itemize}

\section{Customização de Habilidades}

Cada jogador pode criar habilidades únicas, com a aprovação do Mestre.

\chapter{Mecânica do Jogo}

\section{Sistema de Resolução de Ações}

Todas as ações usam um sistema d100. Cada habilidade permite um teste de d100, somando os valores adequados. O sucesso depende do valor final:
\begin{itemize}
    \item Se o resultado for \textbf{<= ao valor da habilidade + bônus}, é um sucesso.
    \item Se o dado for \textbf{100}, a ação funciona para atividades não sobrenaturais.
    \item Se o dado for \textbf{1}, ocorre uma falha crítica e não é possível realizar a ação.
    \item Se a soma for 99, acontece um quase sucesso.
\end{itemize}

As dificuldades comuns são:
\begin{itemize}
    \item \textbf{Ação Fácil}: 75\% ou mais. Exemplo: abrir uma tampa de vidro com força extra.
    \item \textbf{Ação Média}: 50\%. Exemplo: escrever uma carta sem erros.
    \item \textbf{Ação Difícil}: 25\%. Exemplo: realizar um curativo avançado.
    \item \textbf{Ação Extrema}: menos de 10\%. Exemplo: tocar uma peça de piano sem erros.
\end{itemize}

Exemplo: O \textbf{Detetive} tenta abrir uma porta antiga (75\%). Com Força 20, ele precisa de um valor <= 95. Rolando 1d100 e obtendo 67, ele consegue abrir a porta.

\chapter{Combate e Conflitos}

\section{Sistema de Combate}

O combate usa d100, baseado em habilidades de combate.
\begin{itemize}
    \item \textbf{Ataque}: Se o valor rolado for <= ao valor da habilidade de combate + bônus, o ataque acerta.
    \item \textbf{Dano}: O dano é determinado pelo Mestre, podendo variar conforme a arma e a situação.
\end{itemize}

\section{Exemplo de Combate}

O \textbf{Detetive} enfrenta um \textbf{fantasma} com um revólver com balas de chumbo. Ele tem 30 de habilidade de combate e rola um d100. Ao obter 79, o ataque acerta. Como o fantasma é uma entidade intangível, ele só recebe dano com armas especiais, então o ataque não causa dano.

\section{Resistência Física}

Cada personagem começa com \textbf{100 pontos de resistência}. Quando chega a 0, o personagem fica inconsciente e em risco de morte.

\chapter{Super Poderes e Conhecimento Arcano}

\section{Conhecimento Arcano}

Poderes sobrenaturais podem ser aprendidos com contato a textos e entidades. O uso é como habilidades, mas falhas podem ter efeitos adversos.

\section{Super Poderes}

Ao encontrar entidades sobrenaturais, os personagens podem ganhar superpoderes. Os efeitos podem ser desconhecidos pelo jogador, revelando-se na narrativa.

\chapter{Sugestão de Monstros}

\section{Monstros}

\subsection{Bruxo}

Um bruxo poderoso que usa conhecimento arcano e possui habilidades como feitiçaria, invocação e manipulação psicológica.

\begin{itemize}
    \item \textbf{Força}: 15
    \item \textbf{Destreza}: 30
    \item \textbf{Constituição}: 40
    \item \textbf{Inteligência}: 90
    \item \textbf{Empatia}: 60
    \item \textbf{Beleza}: 200
\end{itemize}

\subsection{Fantasma}

O fantasma é uma entidade que assombra locais, invisível e imune a ataques físicos comuns.

\begin{itemize}
    \item \textbf{Força}: 0 (intangível)
    \item \textbf{Destreza}: 80 (movimento invisível)
    \item \textbf{Constituição}: Não aplicável
    \item \textbf{Inteligência}: 50
    \item \textbf{Empatia}: 10
        \item \textbf{Beleza}: 30
\end{itemize}

\subsection{Zumbi}

Um morto-vivo lento e irracional, buscando instintivamente atacar.

\begin{itemize}
    \item \textbf{Força}: 40
    \item \textbf{Destreza}: 10
    \item \textbf{Constituição}: 70
    \item \textbf{Inteligência}: 5
    \item \textbf{Empatia}: 0
     \item \textbf{Beleza}: 0
\end{itemize}

\subsection{Dragão}

Criatura poderosa e inteligente, com habilidades mágicas.

\begin{itemize}
    \item \textbf{Força}: 90
    \item \textbf{Destreza}: 60
    \item \textbf{Constituição}: 100
    \item \textbf{Inteligência}: 80
    \item \textbf{Empatia}: 30
    \item \textbf{Beleza}: -100 (aterrorizante)
\end{itemize}

\subsection{ET de Varginha}

Entidade alienígena com habilidades tecnológicas e desconhecidas.

\begin{itemize}
    \item \textbf{Força}: 10
    \item \textbf{Destreza}: 20
    \item \textbf{Constituição}: 15
    \item \textbf{Inteligência}: 80
    \item \textbf{Empatia}: 40
    \item \textbf{Beleza}: 20
\end{itemize}

\subsection{Azathoth}

Entidade cósmica de caos e destruição, impossível de ser combatida.

\begin{itemize}
    \item \textbf{Força}: Infinita
    \item \textbf{Destreza}: 100 (onipresente)
    \item \textbf{Constituição}: Infinita
    \item \textbf{Inteligência}: Desconhecida
    \item \textbf{Empatia}: Nenhuma
    \item \textbf{Beleza}: Definida pelo NPC a cada instante
\end{itemize}

\end{document}
