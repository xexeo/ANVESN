 \documentclass[a4paper,12pt]{book}
\usepackage[utf8]{inputenc}
\usepackage{geometry}
\geometry{a4paper, margin=1in}
\setcounter{secnumdepth}{3}
\setcounter{tocdepth}{2}

\usepackage{modernrules}

\title{Manual de Regras Simples para RPG: ANVESN}
\author{Baseado em The Laundry Files e Call of Cthulhu}
\date{}

\begin{document}

\maketitle

\tableofcontents

\chapter{Introdução}

A Agência Nacional de Vigilância de Eventos Sobrenaturais (ANVESN) é uma organização super secreta, destinada ao monitoramento e contenção de fenômenos sobrenaturais e ameaças extraordinárias à segurança nacional. Inspirado por sistemas simples e intuitivos, este RPG oferece uma experiência acessível, com mecânicas de insanidade e habilidades agrupadas em categorias. 

\section{Objetivo do Jogo}

Os jogadores representam agentes da ANVESN, enfrentando entidades sobrenaturais, investigando atividades paranormais e lidando com os riscos físicos e psicológicos que essas missões impõem.

Os jogadores são sempre personagens humanos.

\chapter{Criação de Personagem}

\section{Atributos Básicos: Definindo a Capacidade Física}

Nesse jogo toda a capacidade ligada ao intelecto e emoções são realizadas pelo jogador. 

Os atributos físicos, porém, são do personagem.

Os atributos são gerados com os 3 maiores dados jogando 4D6 (4D6-L). O jogador deve jogar 5 conjuntos de 4D6, escolher os 4 melhores resultados e colocar como deseja em seus atributos básicos.

Os atributos básicos são:

\begin{itemize}
\item \textbf{Força} Representa a capacidade física do personagem para levantar, carregar e mover objetos, bem como a potência de seus ataques corpo a corpo. Um personagem com alta Força é mais eficaz em combates físicos e atividades que exigem esforço muscular. Esse atributo influencia tanto o dano quanto a eficácia ao usar armas físicas e nas atividades que requerem grande força, como abrir portas pesadas ou escalar superfícies íngremes.
\item \textbf{Destreza} Este atributo reflete a agilidade, velocidade e precisão dos movimentos do personagem. Personagens com alta Destreza têm vantagens em ações que exigem precisão, como furtividade, ataques à distância e esquivas em combate. A Destreza também é essencial para tarefas que envolvem coordenação motora fina e reflexos rápidos, como manipular objetos delicados, escapar de armadilhas e executar movimentos acrobáticos.

\item \textbf{Constituição} A Constituição representa a resistência física e a saúde geral do personagem. Personagens com alta Constituição têm maior vitalidade e podem resistir melhor a efeitos debilitantes, como doenças, venenos e fadiga. Esse atributo também afeta a quantidade de pontos de vida, aumentando a capacidade de suportar danos em combate e a longevidade em situações de sobrevivência.

\item \textbf{Beleza} Embora não esteja diretamente ligada à capacidade física, a Beleza é um atributo que representa a aparência física do personagem e sua capacidade de influenciar outros apenas pela presença. É considerada apenas a Beleza física. Poderes de persuação e similares associados ao carisma devem ser usados como habilidades.

\end{itemize}

\section{Habilidades Naturais}

Cada personagem escolhe 5 habilidades naturais da pessoa, que representam habilidades ligada a experiência e história de vida.

\section{Definindo a Formação}

Cada personagem escolhe uma formação profissional, que determinará suas habilidades principais e complementares. Abaixo estão algumas opções de formação sugeridas:

\begin{itemize}
    \item \textbf{Cientista}: Especialista em conhecimento teórico e técnico.
    \item \textbf{Investigador}: Focado em habilidades de percepção e raciocínio.
    \item \textbf{Combatente}: Treinado em habilidades físicas e de combate.
    \item \textbf{Operador de Campo}: Mistura habilidades práticas e de improviso.
\end{itemize}

\section{Escolhendo Habilidades}

Cada personagem pode escolher \textbf{5 habilidades principais} e \textbf{5 habilidades complementares}, baseando-se na formação escolhida. As habilidades são agrupadas em três categorias principais:

Todas as habilidades tem que ser passíveis de serem realizadas por um ser humano \textit{baunilha}, sem nenhum poder especial.

\begin{itemize}
    \item \textbf{Habilidades Físicas} (ex.: Luta, Esquiva, Atletismo)
    \item \textbf{Habilidades de Conhecimento} (ex.: Ocultismo, Ciências, História)
    \item \textbf{Habilidades Práticas} (ex.: Investigação, Tecnologia, Primeiros Socorros)
\end{itemize}

\subsection{Customização de Habilidades}

Cada jogador pode criar habilidade únicas, de acordo com a aprovação do Mestre, que combine com seu personagem e formação.

\chapter{Mecânica do Jogo}

\section{Sistema de Resolução de Ações}

As ações são resolvidas com um sistema simples de dados, onde cada habilidade permite um teste com 1D100. O jogador joga o dado e soma a chance de realizar, o valor do dado e suas habilidade e capacidades físicas adequadas. Se for alcançado o número 100, a ação funcionou. O número 100 nos dados sempre funciona para atividades não sobrenaturais.

Se a soma for 99, ação quase funcionou e deve ter algum efeito narrativo. 

Se o dado for 1, é uma falha crítica. O DM deve determinar os efeitos, considerando a narrativa e as características do personagem.

A chance de uma ação dar certo depende da dificuldade da ação. As dificuldades mais usadas são:

\begin{itemize}
    \item Uma ação que qualquer ser humano pode fazer naturalmente com facilidade: 75\% de chance de dar certo. Exemplo: abrir uma tampa de vidro de conserva que está presa.
    \item Uma ação que exige normalmente alguma prática ou conhecimento. Exemplo: escrever uma carta sem erros de ortografia. Chance básica: 50\%.
    \item Uma ação que exige treinamento profissional: 25\%. Por exemplo, dar pontos em um corte profundo.
    \item Ações que exigem um \textit{virtuose}, como tocar uma peça de piano sem erros: menos de 10\%.
\end{itemize}

Por exemplo, abrir a tampa do vidro de conserva exige 75\%, mas o jogador pode somar sua força (ou sua destreza). Supondo que seja 15, o total é 90\%. Então qualquer número maior que 10 garantiria a abertura.




\begin{itemize}
    \item \textbf{Sucesso}: Em habilidades principais, rolar um total de 7 ou mais é um sucesso. Para habilidades complementares, o valor precisa ser 5 ou mais.
    \item \textbf{Falha}: Resultados abaixo dos valores indicados são falhas.
\end{itemize}

\subsection{Testes Difíceis}

Para ações especialmente complexas, o Mestre pode aumentar a dificuldade em 2 pontos, tornando o sucesso com habilidades principais um total de 9, e habilidades complementares um total de 7.

\section{Aprendizado com Sucesso e Falha}

Sempre que um personagem tiver sucesso ou falha ao tentar uma ação, ele recebe \textbf{1 ponto de experiência} para aquela habilidade específica. A cada \textbf{5 pontos de experiência} em uma habilidade, o jogador pode adicionar um bônus de +1 aos testes daquela habilidade.

\section{Ganho de Experiência e Evolução}

Ao realizar uma ação, o personagem pode ou não adquirir pontos de experiência com base na distância entre o resultado da rolagem e o valor necessário para o sucesso. Esses pontos representam aprendizado em relação à habilidade utilizada, seja a ação bem-sucedida ou não. O ganho de experiência ocorre conforme as seguintes regras:

\begin{enumerate}
    \item \textbf{Sucesso Perfeito (Exatamente o Valor Necessário)}: Caso o resultado da rolagem seja exatamente igual ao valor necessário para o sucesso, o personagem não ganha experiência. A ação foi realizada exatamente conforme o esperado, mas não houve aprendizado adicional.

    \item \textbf{Sucesso com Excedente ou com Aproximação (Acima ou Próximo do Valor Necessário)}:
    \begin{itemize}
        \item \textbf{Acima do Valor Necessário}: Se o resultado da rolagem foi superior ao valor necessário para o sucesso, mas ainda dentro de uma margem de até 2 pontos a mais, o personagem ganha \textbf{1 ponto de experiência}. Ele dominou a ação e ampliou seu entendimento sobre a habilidade.
        \item \textbf{Abaixo, mas Próximo do Valor Necessário}: Caso o resultado da rolagem seja menor que o valor necessário, mas dentro de uma margem de até 2 pontos abaixo, o personagem também ganha \textbf{1 ponto de experiência}, indicando que quase atingiu o sucesso e aprendeu com a tentativa.
    \end{itemize}
    
    \item \textbf{Distante do Valor Necessário (Fracasso Extremo ou Sucesso Excessivo)}: Se o valor da rolagem está mais de 2 pontos abaixo ou acima do valor necessário, o personagem não ganha experiência. Esse resultado indica que a tentativa foi tão distante do objetivo, ou tão além do necessário, que não gerou aprendizado relevante.
\end{enumerate}

\subsection{Evolução Adicional das Habilidades}

Sempre que um personagem acumular \textbf{5 pontos de experiência} em uma habilidade específica, ele pode adicionar \textbf{+1} como bônus em futuras rolagens dessa habilidade, representando seu progresso. Esse ganho reflete o aprimoramento do personagem, aumentando gradativamente sua chance de sucesso em tentativas futuras.




\chapter{Insanidade e Efeitos Psicológicos}

\section{Mecânica de Insanidade}

Ao lidar com eventos sobrenaturais, o personagem está sujeito a efeitos psicológicos que podem levar à insanidade.

\begin{itemize}
    \item \textbf{Teste de Sanidade}: Sempre que um personagem enfrentar uma situação aterradora, ele deve realizar um teste de sanidade, rolando 1d6.
    \item \textbf{Resultado do Teste}: Um resultado de 3 ou menos indica que o personagem sofreu um abalo mental. Marque \textbf{1 ponto de insanidade}.
    \item \textbf{Acúmulo de Pontos de Insanidade}: Ao acumular 5 pontos de insanidade, o personagem entra em uma condição temporária de loucura.
\end{itemize}

\section{Consequências da Insanidade}

Ao atingir 5 pontos de insanidade, o personagem precisa rolar 1d6 para determinar a duração da condição de loucura. Durante essa fase, o Mestre determina as ações do personagem, refletindo os efeitos da insanidade em suas escolhas.

\chapter{Combate e Conflitos}

\section{Sistema de Combate}

O combate é resolvido com base nas habilidades físicas do personagem.

\begin{itemize}
    \item \textbf{Ataque}: O jogador rola um teste com 2d6 usando sua habilidade de combate. Um sucesso indica que o golpe acertou o alvo.
    \item \textbf{Dano}: O dano é determinado pelo Mestre e depende do tipo de arma ou ataque utilizado, podendo ser ajustado conforme o contexto.
\end{itemize}

\section{Resistência Física}

Cada personagem começa com \textbf{10 pontos de resistência}. Sempre que um personagem sofre dano, ele perde pontos de resistência. Ao atingir 0 pontos, o personagem está inconsciente e em risco de morte.

\chapter{Progresso e Evolução dos Personagens}

\section{Ganho de Experiência e Evolução}

Ao final de cada sessão, o Mestre pode conceder pontos de experiência adicionais aos jogadores, com base na performance e na resolução das missões. Os jogadores podem usar esses pontos para melhorar suas habilidades ou adicionar uma nova habilidade complementar, com aprovação do Mestre.

\chapter{Super Poderes e Conhecimento Arcano}

\section{Conhecimento Arcano}

Considera-se conhecimento arcano o uso de magias, feitiçarias, invocações, etc.

Ao entrar em contato com livros e personagens com poderes arcanos, os personagens humanos podem aprender a usá-los. 

O uso de poderes arcanos acontecem da mesma forma que as habilidades, porém a falha pode trazer resultados adversos, principalmente quando for por valores muito grandes.

Os poderes arcanos exigem estudos, e o próprio estudo pode ter efeito adverso.

\textbf{Os personagens humanos, tendo acesso a documentos arcanos, podem escolher os que vão estudar ou não.}


\section{Super Poderes}

Ao entrar em contato com entidades sobrenaturais, os personagens humanos podem ganhar super poderes. Esses super poderes são determinados pelas aventuras, descrições de NPC ou mesmo pelo mestre do jogo e seguem as mesmas regras das habilidades comuns.

\textbf{Os personagens humanos \textbf{nunca} podem escolher os próprios super poderes.}

Algums super poderes podem possuir efeitos colaterais indesejados, que não serão avisados ao jogador, mas podem ser tratados em aberto, por meio da narrativa. Eles também podem ser tratados secretamente pelo mestre, sem efeitos de narrativa, porém se espera que o jogador possa perceber o que acontece, pelo menos com a repetição do efeito.

\end{document}