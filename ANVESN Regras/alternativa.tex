\documentclass[a4paper,12pt]{book}
\usepackage[utf8]{inputenc}
\usepackage{geometry}
\geometry{a4paper, margin=1in}
\setcounter{secnumdepth}{3}
\setcounter{tocdepth}{2}

\usepackage{modernrules}

\title{Manual de Regras Simples para RPG: ANVESN}
\author{Baseado em The Laundry Files e Call of Cthulhu}
\date{}

\begin{document}

\maketitle

\begin{center}
\newpage
\vspace*{\fill}
\includegraphics[scale=.9]{imagens/ANVESN_LOGO.png}
\vspace*{\fill}
\newpage
\end{center}


\tableofcontents

\chapter{Introdução}

A Agência Nacional de Vigilância de Eventos Sobrenaturais (ANVESN) é uma organização super secreta, destinada ao monitoramento e contenção de fenômenos sobrenaturais e ameaças extraordinárias à segurança nacional. Inspirado por sistemas simples e intuitivos, este RPG oferece uma experiência acessível, com mecânicas de insanidade e habilidades agrupadas em categorias.

\section{Objetivo do Jogo}

Os jogadores representam agentes da ANVESN, enfrentando entidades sobrenaturais, investigando atividades paranormais e lidando com os riscos físicos e psicológicos que essas missões impõem.

\chapter{Criação de Personagem}

\section{Atributos Básicos: Definindo a Capacidade Física}

Os atributos físicos do personagem são definidos com a rolagem de dados de 4D6-L (3 maiores valores de 4D6) para determinar Força, Destreza, Constituição e Beleza.

Nesse jogo não há atributos intelectuais, já que estes são criados pelo jogador. Quando necessário são usadas habilidades.

\section{Habilidades}

As habilidades do personagem vem de três fontes: sua natureza humana, sua formação e eventos do jogos.

O jogador deve escolher para o personagem uma \textbf{profissão} de qualquer nível educacional.


As habilidades podem ser em dois níveis básicos: normal,   e especialista (75\% de bônus):

\begin{itemize}
\item \textbf{Não escolhidas}, que podem ser invocadas a qualquer momento e dão um bônus de até 50\%.
\begin{itemize}
\item \textbf{naturais do ser humano}, como correr, jogar pedras, etc. Essas habilidades não precisam ser escolhidas. O bônus básico é 25\%.
\item \textbf{profissionais} que são esperadas de um profissional formado em um nível razoável dentro de sua especialidade, como diagnosticar uma gripe para um médico, ou auscultar uma pneumonia. Essas dão 50\% de bônus.
\end{itemize}
\item \textbf{escolhidas}, que significam especialidades perseguidas pelo jogador. O jogador deve escolher  especialidades, entre as naturais e profissionais, e receberá inicialmente um bônus de 75\% ao realizá-las.
\item \textbf{estudadas} que precisam ser obtidas durante o jogo perante um processo de estudo, mediado por exigências dos cenários ou do mestre do jogo e que não tem bônus pré-definido.
\item \textbf{super-humanas}, que são adquiridas durante o jogo por outros meios, como imortalidade por se tornar vampiro.
\end{itemize}

Não há limite ao que são as habilidades, mas se forem muito gerais, o mestre poderá reduzir seu efeito.


Escolhas:
\begin{itemize}
\item escolher uma profissão técnica ou superior, ou mesmo doutorado e mestrado, para um bônus geral de 50\%.
\item escolher 5 habilidades específicas da profissão para um bônus de 75\%.
\item escolher 10  habilidades gerais, como correr, convencer, \textit{pick-pockets}, para um bônus de 50\%.
\item escolher 2  habilidades gerais, como correr, convencer, \textit{pick-pockets}, para um bônus de 75\%.
\item escolher 5 habilidades específicas que definem seu personagem como um agente da ANVESN, a um nível de especialista (75\% de bônus).
\end{itemize}


Exemplo: o agente precisa fotografar uma cena do crime, mas nunca estudou fotografia. Porém qualquer ser humano pode fotografar com seu celular. Logo ele receber 25\% de bônus, mais a sua destreza (15\%), atingindo 40\% de bônus. O mestre pode dizer que é uma atividade muito fácil para informações gerais (mais 25\% de bônus) e considerar, sigilosamente, que é difícil que a foto permita indentificar algo importante (25\% de penalidade). O jogador rola o dado e tira 60, logo ele soma 125 para informações gerais, mas apenas 75\% na informação específica, não recebendo a informação extra.


\chapter{Mecânica do Jogo}

\section{Sistema de Resolução de Ações}

Todas as ações são resolvidas com 1D100. Os jogadores devem somar o valor do dado ao atributo relevante e às habilidades do personagem para comparar com um valor-alvo pré-definido de dificuldade.

Toda ação deve alcançar pelo menos 100 pontos para acontecer.

Para executar uma ação são contados:
\begin{itemize}
\item Um valor de dificuldade da ação dado pelo cenário ou arbitrado pelo mestre, de forma negativa
\item Uma característica física do personagem, caso seja aplicado, de forma positiva
\item Uma habilidade do personagem, de forma positiva
\item Uma penalidade para uma atividade sem treino, derterminada pela narrativa, por exemplo, atirar sem treino pode ter uma penalidade de 50\% em uma situação de tensão e uma penalidade extra de 50\% a partir do segundo tiro seguido sem mira.
\item Um possível bônus pela qualidade da narrativa, de forma positiva, incluindo ações como mirar antes de atirar ou se agachar ao entrar em um lugar de forma escondida.
\item O valor de 1D100.
\end{itemize}


A soma desses valores deve ser mais de 100 pontos. Qualquer ação humana funciona com o valor 100 no dado, qualquer ação, humana ou não, falha com o valor 1 no dado.

\chapter{Combate e Conflitos}

\section{Sistema de Combate}

O combate usa um sistema simplificado de D100:
\begin{itemize}
    \item \textbf{Ataque}: O jogador rola 1D100 para atacar. Um valor igual ou maior que 100 após as modificações.
    \item \textbf{Dano}: O dano é fixo, baseado em armas ou habilidades do personagem, e é determinado pelo Mestre, simplificando cálculos durante o jogo.
\end{itemize}


\section{Resistência Física}

Cada personagem começa com 100 pontos de resistência. Pontos de resistência são reduzidos conforme o dano sofrido. Ao atingir 0 pontos, o personagem está inconsciente.

O mestre pode declarar o personagem insconsciente por causa do tipo de dano, como pancada na cabeça. Todos os efeitos de narrativa devem ser aplicados.

\chapter{Insanidade e Efeitos Psicológicos}

\section{Mecânica de Insanidade}

Todo personagem começa são, com 100 pontos de sanidade.

Sempre que um personagem enfrentar eventos sobrenaturais, ele deve fazer um teste de sanidade contra o nível de insanidade do evento (de 1 a 100), rolando 1D100. Esse valor deve ultrapassar o nível de insanidade.

Caso o personagem não passe, o que faltou para ele passar é o que perde de sanidade.



Se perder mais de 50\% dos pontos com que o evento, o personagem fica temporariamente insano.

Se sua sanidade chegar a zero, o personagem fica insano.

O personagem pode fazer tratamentos para melhorar sua sanidade.



\chapter{Ganho de Experiência e Evolução}

\section{Ganho de Experiência}

Ao falhar em uma ação, se a falha for por menos de 10\%  dos pontos necessários, há uma falha significativa onde o jogador aprendeu algo. Se foi sucesso, o jogador ganha experiência.

Para cada ação bem-sucedida ou falha significativa, o personagem ganha 1 ponto  na habilidade relevante imediatamente.

Porém, em um mesmo dia, o personagem só pode melhorar 1 ponto em uma habilidade.

Isso vale para todas as habilidades que o jogador executar ao jogar, mesmo as que não escolheu, e deve ser registrado.

Por exemplo, se o jogador decidir correr para pegar o trem, e consegue pegar o trem, ele ganha 1 ponto em correr.


\chapter{Conhecimento Arcano e Super Poderes}

\section{Conhecimento Arcano}

O contato com conhecimento arcano permite que os personagens aprendam habilidades místicas com o Mestre. Falhas em rolagens podem acarretar consequências adversas.

\section{Infecção por Super Poderes}

É possível que em certas situações os personagens ganhem, de forma temporária ou permanente, super poderes.

\end{document}
