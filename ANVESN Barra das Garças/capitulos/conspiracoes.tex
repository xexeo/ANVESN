\chapter{Conspirações em Barra das Garças}

O capítulo apresenta algumas das conspirações que permeiam Barra das Garças. Cada seção descreve uma teoria, os personagens envolvidos e as possíveis conexões com eventos sobrenaturais. Esse capítulo oferece suporte narrativo para os jogadores explorarem mistérios ocultos e investigarem as motivações de figuras influentes da cidade.

\section{A Conspiração da Cidade Perdida de Fawcett}

A primeira e mais famosa das teorias conspiratórias de Barra das Garças gira em torno da expedição do coronel \pindex{Percy Fawcett} e seu desaparecimento em 1925. A teoria sustenta que o desaparecimento de \pindex{Fawcett} foi resultado de uma tentativa de alcançar uma cidade perdida chamada Z, que estaria em uma dimensão paralela no Intraterra.

\subsection{Os Intraterrenos e o Destino de Fawcett}
A teoria sugere que \pindex{Fawcett} teria encontrado uma passagem para o Intraterra e que os habitantes dessa dimensão o capturaram para impedir que seu segredo fosse revelado. A existência de seu neto, \pindex{Edgar Fawcett}, reforça a crença de que Fawcett alcançou a cidade perdida, mas que nunca conseguiu retornar. Edgar, agora guardião das tradições do Intraterra, busca proteger o local de qualquer explorador que tente repetir a façanha de seu avô.

\subsection{Relatos e Evidências}
Registros da biblioteca e arquivos de ``O Araguaia'' contêm relatos detalhados sobre o desaparecimento de Fawcett e documentos de testemunhas locais que relataram luzes misteriosas e sons incomuns vindos das cavernas. O mapa incompleto do Intraterra encontrado nos arquivos do jornal sugere que Fawcett pode ter desenhado um caminho para o coração das montanhas antes de desaparecer.

\section{A Conspiração do Discoporto e do DTCEA-BW}

Esta teoria gira em torno do Destacamento de Controle do Espaço Aéreo de Barra do Garças (DTCEA-BW) e do suposto ``discoporto'' alienígena. A conspiração sugere que o destacamento foi instalado na região para monitorar não apenas o tráfego aéreo, mas também as atividades extraterrestres, incluindo visitas regulares ao discoporto construído nos arredores de Barra das Garças.

\subsection{A Base Militar e os Avistamentos}
O \pindex{Coronel Vasconcelos} e a \pindex{Tenente Freitas} seriam figuras centrais em um encobrimento de avistamentos de luzes e objetos não identificados na região. Segundo rumores, relatórios confidenciais detalham eventos incomuns, mas são mantidos fora dos registros oficiais. A conspiração afirma que esses avistamentos representam tentativas de contato extraterrestre e que o discoporto serve como ponto de referência para essas entidades.

\subsection{O Papel de Bento e os Intraterrenos}
\pindex{Bento Silva}, em aliança com os intraterrenos, supostamente monitora as atividades do DTCEA-BW para evitar que os militares descubram a presença de portais interdimensionais. A teoria propõe que Bento também estaria usando o encobrimento para manipular as opiniões e interesses dos moradores, reforçando as histórias de intraterrenos e alienígenas para desviar a atenção de seus planos com os seres subterrâneos.

\section{A Conspiração do Prefeito e a Proteção Oculta}

A teoria da proteção oculta de \pindex{Helena Souza}, esposa do prefeito Antônio de Souza, está entre as mais intrigantes. A conspiração sustenta que Helena é filha de uma figura sobrenatural, protegida por um espírito maternal que vigia seus passos.

\subsection{O Espírito Maternal e a Tradição Familiar}
O espírito que assombra o cemitério de Barra das Garças é supostamente a mãe de Helena, que morreu quando ela era criança. Esse espírito permanece na cidade para proteger a filha e seu papel como uma liderança influente, em uma tentativa de manter a ordem na cidade. A presença de Helena nas atividades da prefeitura, combinada com a discrição do prefeito, sustenta a teoria de que a proteção sobrenatural influencia a governança local.

\subsection{A Conexão com o Paranormal}
Personagens que investigarem os registros históricos e os arquivos da prefeitura podem encontrar evidências que conectam a presença de Helena à proteção de espíritos obsessores, incluindo o interesse do prefeito em manter segredo sobre as origens místicas de sua esposa. Esses registros também mencionam as ligações de Helena com o cemitério, reforçando a ideia de que há uma conexão espiritual entre ela e as forças ocultas da cidade.

\section{A Conspiração da Infiltração Alienígena}

Finalmente, a conspiração de que o ET e outros extraterrestres têm um papel ativo em Barra das Garças. Segundo essa teoria, a chegada do ET à cidade não foi acidental; ele estaria em uma missão para estudar a interação entre os humanos e os intraterrenos, coletando informações para uma possível aliança com os seres do Intraterra.

\subsection{A Missão do ET e a Conexão com Bento}
De acordo com a teoria, Bento teria recebido informações sobre a presença do ET antes de seu pouso e teria elaborado planos para capturá-lo, acreditando que uma aliança com os intraterrenos traria poder e recursos. Relatos do discoporto e avistamentos de luzes misteriosas reforçam essa teoria, sugerindo que a cidade serve como ponto de encontro para espécies extraterrestres interessadas nos recursos e nos segredos do Intraterra.

\subsection{Os Intraterrenos e a Aliança Interdimensional}
Os intraterrenos, conhecidos por viverem em cidades ocultas sob a superfície, estariam dispostos a cooperar com o ET, desde que ele não representasse uma ameaça. A teoria sugere que essa relação pode trazer benefícios mútuos, e que Bento quer usar essas alianças para expandir seu controle sobre as cavernas da região e as atividades do discoporto.

Cada uma dessas conspirações oferece aos personagens pistas que podem seguir e eventos que podem explorar, enriquecendo a trama do jogo e proporcionando um enredo complexo e dinâmico.

\section{Conspiração dos Intraterrenos}
Bento Silva, um influente líder local, lidera uma conspiração secreta com os intraterrenos, seres misteriosos que habitam cavernas subterrâneas na região de Barra das Garças. Esse grupo trabalha em colaboração com Bento para proteger a cidade de supostas ameaças externas e, ao mesmo tempo, manipular a opinião pública e manter o controle da cidade. Reuniões secretas e trocas de recursos ocorrem em cavernas isoladas, nas quais os intraterrenos fornecem informações em troca de itens da superfície.

\subsection{Personagens Envolvidos}
\begin{itemize}
    \item \textbf{Bento Silva}: Líder da conspiração, tem interesses ocultos em manipular as narrativas sobre os intraterrenos para manter sua influência.
    \item \textbf{D. Lurdinha}: Testemunha que, em sua infância, presenciou fenômenos sobrenaturais próximos à floresta.
    \item \textbf{Tonico, o Pescador}: Suspeita da relação de Bento com os intraterrenos e é cético quanto aos reais objetivos de Bento.
    \item \textbf{Dona Cláudia, Secretária da Prefeitura}: Observa Bento e suspeita de suas atividades, mantendo registros de eventos sobrenaturais ligados à prefeitura.
    \item \textbf{Prefeito Antônio de Souza}: Embora não diretamente envolvido, o prefeito lida com os boatos e a pressão política derivada das atividades de Bento e dos mistérios sobrenaturais.
\end{itemize}

\subsection{Pistas Associadas}
\begin{itemize}
    \item \textbf{Convite para Reunião Secreta}: Panfleto sobre uma reunião organizada por Bento e os intraterrenos, discutindo a “proteção” da cidade.
    \item \textbf{Broche de Intraterrenos}: Um objeto usado pelos aliados de Bento, simbolizando sua aliança com os seres subterrâneos.
    \item \textbf{Notas sobre a Sala de Monitoramento}: Documentos que descrevem encontros e trocas de recursos entre humanos e intraterrenos.
\end{itemize}

\section{Encobrimento e Manipulação do Misticismo Local}
Miguel Rocha, vereador e proprietário do jornal “O Araguaia,” lidera uma conspiração para manipular o turismo em Barra das Garças. Ele utiliza o misticismo local e as lendas sobre fenômenos sobrenaturais para atrair visitantes, explorando histórias de avistamentos e assombrações. Com isso, Miguel assegura sua influência política e econômica na cidade.

\subsection{Personagens Envolvidos}
\begin{itemize}
    \item \textbf{Miguel Rocha}: Vereador influente e dono do jornal “O Araguaia.” Ele usa o jornal para divulgar histórias e manipular o turismo.
    \item \textbf{Jorge Almeida, Jornalista}: Jornalista veterano que raramente questiona as pautas impostas por Miguel, contribuindo para a perpetuação das lendas.
    \item \textbf{Maria Clara Reis, Jornalista}: Idealista e opositora, ela mantém um blog anônimo, “Voz do Araguaia,” onde publica reportagens que Miguel rejeita.
\end{itemize}

\subsection{Pistas Associadas}
\begin{itemize}
    \item \textbf{Pastas de Edições Antigas}: Contêm histórias e lendas manipuladas por Miguel para atrair turismo.
    \item \textbf{Edição Perdida de “O Araguaia”}: Edição censurada mencionando a relação entre figuras políticas e o fenômeno dos intraterrenos.
    \item \textbf{Recortes sobre Fenômenos Sobrenaturais}: Compilação de avistamentos de OVNIs e lendas locais, usados para manter o misticismo da cidade.
\end{itemize}

\section{O Assassinato de Joaquim Brandão}
Joaquim Brandão, um jornalista local, foi assassinado enquanto investigava as atividades ilegais de Bento Silva, que incluiam alianças com os intraterrenos. Bento eliminou Joaquim para proteger seus segredos, incluindo evidências de reuniões clandestinas e troca de informações. No entanto, o espírito de Joaquim permanece no cemitério, tentando revelar a verdade.

\subsection{Personagens Envolvidos}
\begin{itemize}
    \item \textbf{Joaquim Brandão, Jornalista}: O fantasma que busca justiça e deseja expor os segredos de Bento.
    \item \textbf{Bento Silva}: Assassino de Joaquim, com interesses em eliminar qualquer ameaça às suas conspirações.
    \item \textbf{Paulo Farias, Escrivão da Delegacia}: Guarda registros antigos e relatórios não concluídos sobre Joaquim e Bento.
\end{itemize}

\subsection{Pistas Associadas}
\begin{itemize}
    \item \textbf{Diário de Joaquim Brandão}: Guardado na biblioteca, revela que Joaquim estava investigando Bento e suas reuniões noturnas.
    \item \textbf{Carta Rasgada}: Encontrada no escritório do jornal, menciona um acordo com “figuras ocultas” e uma advertência anônima a Joaquim.
    \item \textbf{Relatório de Investigação Incompleto}: Na delegacia, menciona locais onde Joaquim suspeitava que Bento realizava reuniões clandestinas.
    \item \textbf{Fotos Reveladoras}: Escondidas no Bar “Lua Cheia,” mostram Bento em reuniões suspeitas com figuras misteriosas.
\end{itemize}


