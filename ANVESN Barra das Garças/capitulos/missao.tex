\chapter{A Missão}

\section{Objetivo: Verificar a Assombração no Cemitério}
A cidade de Barra das Garças, um destino turístico místico e esotérico na região Centro-Oeste do Brasil, é conhecida por suas paisagens naturais, como a Serra do Roncador e a confluência dos rios Araguaia e Garças. Mas é o Cemitério Municipal que tem chamado a atenção da ANVESN nas últimas semanas. Diversos relatos de avistamentos de uma figura espectral circulam entre os moradores, especialmente durante a noite. Testemunhas afirmam que o fantasma, possivelmente de uma mulher, aparece flutuando entre as lápides, emitindo uma leve luminescência azulada e pronunciando palavras incompreensíveis.

A missão inicial dos agentes é investigar essas ocorrências no cemitério, determinar a veracidade dos relatos e, se confirmado o fenômeno, conter a presença sobrenatural para evitar um possível aumento no turismo sensacionalista ou na histeria local. A ANVESN já identificou várias testemunhas dispostas a fornecer depoimentos sobre a aparição.

\section{Briefing da Cidade: Barra das Garças}
Barra das Garças é uma cidade de aproximadamente 61 mil habitantes, conhecida não apenas por sua beleza natural, mas também pela rica cultura folclórica e histórias sobrenaturais. A cidade é famosa por lendas locais sobre portais dimensionais e atividades alienígenas, ganhando reputação como um ponto místico e de avistamento de OVNIs. Com o desaparecimento misterioso do coronel inglês Percy Fawcett em 1925, enquanto explorava a Serra do Roncador em busca da lendária Cidade Z, a cidade tornou-se um lugar ainda mais intrigante para curiosos e místicos.

A cidade conta com diversos estabelecimentos que exploram essa atmosfera, como o jornal sensacionalista ``O Araguaia'' e o Bar ``Lua Cheia'', onde histórias sobre o cemitério e aparições locais são discutidas com entusiasmo. Esse interesse no sobrenatural faz parte da identidade local, mas também traz desafios para a ANVESN em conter possíveis excessos e histeria.

\section{Informações Práticas e Rumores Místicos}
\begin{itemize}
    \item \textbf{Localização do Cemitério}: Situado no perímetro norte da cidade, o cemitério é uma área bastante arborizada e pouco iluminada, especialmente à noite, o que contribui para o clima de mistério. A entrada principal é vigiada por uma guarda noturna, que também relata sons estranhos e aparições de luzes.

    \item \textbf{Horário dos Avistamentos}: Os relatos de aparição do fantasma variam, mas a maioria das testemunhas indica que ele surge entre meia-noite e duas da manhã, período conhecido na cidade como a ``Hora Azul''. Moradores mais velhos dizem que este é o momento em que o véu entre os mundos é mais fino, facilitando manifestações sobrenaturais.

    \item \textbf{Fofoqueira Local, Dona Lurdinha}: Dona Lurdinha, proprietária da venda de quitutes na Praça Central, tem muitos relatos sobre o cemitério e garante que o fantasma é o espírito de uma mulher que perdeu seu amor na Serra do Roncador. Ela afirma que a assombração busca por algo, mas nunca encontra, repetindo a jornada todas as noites. Dona Lurdinha é uma fonte constante de rumores e gosta de alimentar o mistério em suas conversas com os visitantes.

    \item \textbf{Conexão com a Igreja de São Miguel}: Alguns moradores associam o espírito a eventos ocorridos na Igreja de São Miguel, onde, há muitos anos, foram registrados exorcismos e rituais para proteção da cidade. A igreja, localizada próxima ao cemitério, é frequentada por devotos que acreditam que as almas perturbadas do cemitério são atraídas pela energia do local. O padre atual, no entanto, nega qualquer envolvimento direto com o fenômeno e se recusa a comentar o assunto.

    \item \textbf{O Jornal ``O Araguaia'' e sua Cobertura}: ``O Araguaia'' é conhecido por exagerar ou até fabricar histórias para aumentar o turismo místico na cidade. Recentemente, publicou uma matéria detalhada sobre o fantasma do cemitério, associando-o a portais interdimensionais e teorias de conspiração. Essa matéria aumentou a atenção sobre o caso, e a ANVESN suspeita que as descrições exageradas estejam contribuindo para a histeria.

    \item \textbf{Relatos de Luzes Misteriosas na Serra}: Além do cemitério, há frequentes relatos de luzes inexplicáveis na Serra do Roncador, que muitos moradores acreditam estar ligadas ao espírito do cemitério. Algumas teorias populares afirmam que essas luzes guiam almas perdidas de volta ao cemitério, alimentando ainda mais a aura de mistério.

    \item \textbf{Sinais de Atividade Esotérica na Região}: A região tem sido alvo de atividades de grupos esotéricos, que realizam rituais e vigílias à noite, afirmando que o cemitério é um ponto de contato com outras dimensões. Esse movimento atrai curiosos e gera polêmicas locais, pois há aqueles que acreditam que esses grupos despertam energias que deveriam permanecer adormecidas.

\end{itemize}

\section{Ação Inicial da Missão}
Os agentes da ANVESN devem chegar ao cemitério antes do horário crítico de meia-noite, realizando uma varredura inicial para garantir que não haja pessoas ou grupos esotéricos na área. É importante que os agentes estabeleçam contato com a guarda noturna e com eventuais testemunhas nas proximidades para obter mais informações.

Caso o fantasma se manifeste, os agentes devem proceder com medidas de contenção e, se necessário, exorcismo, para selar qualquer presença sobrenatural. A missão primária é observar e relatar, mas a ANVESN também autoriza os agentes a usarem equipamentos especiais de detecção de atividades ectoplásmicas e selos de contenção, caso o fenômeno apresente risco aos cidadãos de Barra das Garças.


