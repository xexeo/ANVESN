\chapter{Memorando de Missão Adicional}
\fbox{
\begin{minipage}{0.9\textwidth}
\footnotesize
\begin{center}
\textbf{Agência Nacional de Vigilância de Eventos Sobrenaturais} 

\textbf{Informações Complementares da Missão}
\end{center}
\hrule
\vspace{0.3cm}

\noindent
\textbf{Destinatário:} Equipe de Operações da Diretoria de Assombrações e Entidades Espirituais \\
\textbf{Data:} 05 de Novembro de 2024 \\
\textbf{Assunto:} Informações Adicionais sobre o Evento Sobrenatural no Cemitério Municipal de Barra das Garças

\vspace{0.3cm}
\noindent
Prezados agentes,

\noindent
Em complemento às instruções operacionais anteriormente designadas, seguem informações adicionais e contexto local relevantes para a missão de verificação e contenção de evento sobrenatural no Cemitério Municipal de Barra das Garças. Essas informações podem auxiliar na compreensão do fenômeno e na interação com moradores e testemunhas locais.

\vspace{0.3cm}
\noindent
\textbf{Informações Adicionais sobre o Evento:}
Diversos relatos de aparições fantasmagóricas foram registrados, sugerindo a presença de uma entidade luminosa que aparece entre meia-noite e duas da manhã, horário conhecido entre os locais como a “Hora Azul”. Testemunhas afirmam que o fantasma, possivelmente uma mulher, emite uma luz azulada e sussurra palavras incompreensíveis. Rumores apontam que a entidade pode estar relacionada com antigas lendas locais, mas não se descarta a hipótese de manipulação esotérica para fins turísticos, devido à cobertura sensacionalista do jornal ``O Araguaia''.

\vspace{0.3cm}
\noindent
\textbf{Contexto Local e Fofoqueiros Regionais:}
\begin{itemize}
    \item \textbf{Dona Lurdinha}, conhecida proprietária de uma venda de quitutes na praça central, afirma que a entidade é o espírito de uma mulher à procura de um amor perdido na Serra do Roncador. Dona Lurdinha poderá fornecer rumores e informações úteis para uma abordagem mais contextual.
    \item \textbf{Conexão com a Igreja de São Miguel}: Moradores associam o fantasma a antigos rituais de proteção realizados na Igreja de São Miguel, próxima ao cemitério. Embora o padre atual negue qualquer ligação, esse local pode apresentar valor simbólico no contexto da operação.
\end{itemize}

\vspace{0.3cm}
\noindent
A equipe deve estar ciente da sensibilidade local ao tratar do tema das aparições, visto que tais fenômenos são parte da cultura popular e frequentemente explorados em benefício do turismo esotérico da região.

\vspace{0.3cm}
\noindent
Atenciosamente, \\
\vspace{0.3cm}
\textbf{Diretoria de Assombrações e Entidades Espirituais} \\

\end{minipage}
}


